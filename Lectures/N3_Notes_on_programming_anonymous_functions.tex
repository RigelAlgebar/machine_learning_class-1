\documentclass[11pt]{beamer}

%START To get MATLAB environment
\usepackage[numbered,framed]{matlab-prettifier}
%\usepackage{filecontents}

\let\ph\mlplaceholder % shorter macro
\lstMakeShortInline"

\lstset{
	style              = Matlab-editor,
	basicstyle         = \tiny \ttfamily,
	escapechar         = ",
	mlshowsectionrules = true,
}

%\renewcommand{\lstlistingname}{Algorithm}% Listing -> Algorithm
\renewcommand{\lstlistingname}{Code}% Listing -> Code
%FINISH To get MATLAB environment

%\usepackage{standalone}
%\graphicspath{{../}{../}}
%\graphicspath{{../../figures/}}

\setcounter{tocdepth}{1}

\usepackage[english]{babel}

\usepackage{amsmath}
\usepackage{amsfonts}
\usepackage{amssymb}
\usepackage{graphicx}
\usepackage{tikz}
\usetikzlibrary{shapes.geometric, arrows}
\tikzstyle{startstop} = [rectangle, rounded corners, minimum width=1.7cm, minimum height=0.7cm,text centered, draw=black, fill=red!30]
\tikzstyle{io} = [trapezium, trapezium left angle=70, trapezium right angle=110, minimum width=1.7cm, minimum height=0.7cm, text centered, draw=black, fill=blue!30]
\tikzstyle{process} = [rectangle, minimum width=1.7cm, minimum height=0.7cm, text centered, draw=black, fill=orange!30]
\tikzstyle{decision} = [diamond, minimum width=1.7cm, minimum height=0.7cm, text centered, text width=1.7cm, draw=black, fill=green!30]
\tikzstyle{arrow} = [thick,->,>=stealth]



\setbeamercovered{transparent}
%\usepackage{enumitem}
%\setlist[itemize]{leftmargin=*}

%\usepackage{mhchem}
%\usepackage[utf8]{inputenc}
%\usepackage[T1]{fontenc}
%\numberwithin{equation}{section}



\author[Jose Mendoza-Cortes]{Prof. Jose L. Mendoza-Cortes}
\title[Machine Learning]{Machine Learning}
%\subtitle{Spring '18}
\institute[]
{\scriptsize  
	Scientific Computing Department, Dirac Science Building \\
	Materials Science and Engineering, High Performance Materials Institute\\
	Florida State University\\
	\href{mailto:jmendozacortes@fsu.edu}{jmendozacortes@fsu.edu}\\[3mm]
	
	Condensed Matter Theory, National High Magnetic Field Laboratory\\%[3mm]
	Florida State University\\	
	\href{mailto:mendoza@magnet.fsu.edu}{mendoza@magnet.fsu.edu}\\[3mm]	
	
	Chemical and Biomedical Engineering \\
	Florida State University | Florida A\&M University | College of Engineering \\
	\href{mailto:mendoza@eng.famu.fsu.edu}{mendoza@eng.famu.fsu.edu}\\[3mm]
	Web: \href{http://mendoza.eng.fsu.edu/}{http://mendoza.eng.fsu.edu/}\\%[1mm]
}  

\date{}
\subject{Theory and Computations in Materials, Chemistry and Physics}

%\usetheme{Berkeley}
%\logo{\includegraphics[scale=0.213]{figures/fsu_logo.png}}
%\addtobeamertemplate{navigation symbols}{}{%
%    \usebeamerfont{footline}%
%    \usebeamercolor[fg]{footline}%
%    \hspace{1em}%
%    \insertframenumber/\inserttotalframenumber
%}

\usetheme{Madrid}
%\usecolortheme{beaver}
%\usecolortheme{orchid}

\newif\ifplacelogo % create a new conditional
\placelogotrue % set it to true
\logo{\ifplacelogo
	\includegraphics[width=0.1\linewidth]{figures/fsu_logo.png}
	\includegraphics[width=0.1\linewidth]{figures/famu_logo.png}
	\includegraphics[width=0.1\linewidth]{figures/maglab_logo.png}
	\fi} % replace with your own command

\definecolor{mycustom}{RGB}{0,0,102}       %102,38,38 %128,0,0
%\definecolor{mycustom}{RGB}{128,0,0}       %102,38,38 %128,0,0
%
%\setbeamercolor{structure}{bg=white, fg=custom}
%\setbeamercolor{caption}{fg=custom}

\definecolor{custom}{cmyk}{1,0.5,0,0.47}       %102,38,38 %128,0,0

\setbeamercolor{structure}{bg=white, fg=custom}
\setbeamercolor{caption}{fg=custom}

\setbeamertemplate{navigation symbols}{} %To remove the navigation symbols

%\setbeamercolor{frametitle}{fg=custom}
%\setbeamercolor{framesubtitle}{fg=custom}
\setbeamercolor{titlelike}{parent=structure,bg=gray!20!white}

\setlength\abovecaptionskip{-3pt}
\setbeamertemplate{caption}{%
	\insertcaptionname\,\insertcaptionnumber:\,\insertcaption
}



\usepackage{hyperref}
\hypersetup{colorlinks=true,
	linkcolor=mycustom,
	urlcolor=mycustom}

\abovedisplayskip=0pt
\belowdisplayskip=0pt


\usepackage{pgfpages}
\pgfpagesuselayout{2 on 1}[letterpaper,%landscape,
border shrink=5mm]
	
	
	\begin{document}
		
		\placelogotrue % turn the logo off \usetheme{Madrid}
		\maketitle
		
		\placelogofalse % turn the logo off

%\section{Lecture 4}

\section{Anonymous Functions and Recursion}
%*****************

\begin{frame}[fragile]
	\frametitle{Overview}
	%\centering
	\vspace{-11pt}
	\begin{minipage}[t]{0.43\linewidth}
		\vspace{-10pt}
		\begin{enumerate}
			\item Review last class 
			\item Anonymous Functions
			\item Recursion
		\end{enumerate}
		\vspace{-1pt}
		\begin{alertblock}{Remember}
			\begin{itemize}
				\item MATLAB is case sensitive!. \\
				\item If you declare a \textbf{variable} \verb|Temp| and then use \verb|temp|, MATLAB will give an error
				\item If you define \textbf{function} \verb|Pressure| but then try to run \verb|pressure| with F5 or run, MATLAB will give you an error (assuming your function needs inputs).
			\end{itemize}
		\end{alertblock}
	\end{minipage}
	\hspace{10pt}
	\begin{minipage}[t]{0.51\linewidth}
		\vspace{-15pt}
		\begin{block}{“Camel Case” convention}
\begin{itemize}
	\setlength\itemsep{-0pt}
	\item This useful to keep track of your code and be organized.	
	\item \textbf{Variables} named with a single word are all lower case.  Example: \verb|position|
	\item \textbf{Variables} named using more than one word, should include capitalization on every word but he first. Examples: \verb|centerPosition|, \verb|centerOfGravity|
	\item \textbf{Function} names should always start with a capital letter starting each word
	Examples: \verb|CalculateCenterPosition|
\end{itemize}	
		\end{block}
	\end{minipage}
	
\end{frame}

%*******************************************
%*******************************************
\section{Review on functions}

\subsection{Early example}
\begin{frame}[fragile]
	\frametitle{\secname: \subsecname}
	\vspace{-2mm}
	\lstinputlisting{N3_Notes_on_programming_anonymous_functions/Pressure.m}
	
	\vspace{-3mm}	
\begin{lstlisting}[caption = {Script}]
thePressureis = Pressure(298,10)
\end{lstlisting}

	\vspace{-2mm}	
	\begin{alertblock}{Remember the difference with local variables}
		Variables defined inside scripts are valid after executing the script (i.e. you will see them in the workspace window, e.g. \verb|298, 10|).\\
		However variables within a function are said to be \textbf{local} and erased after the function is executed (e.g. \verb|R| on this case).
	\end{alertblock}
\end{frame}

%*******************************************
%*******************************************
\begin{frame}[fragile]
	\frametitle{Another way to call assign the output}
	
	\lstinputlisting[caption = {Another way to assign the output}]{N3_Notes_on_programming_anonymous_functions/calling_Pressure.m}
	
	\begin{lstlisting}[caption = {The output for the script above is}]
darklord =
247.7572
	\end{lstlisting}
	
	\begin{exampleblock}{Remember}
		Remember that you can assign the output to whatever variable you choose. On this example we assign the pressure into a variable called \verb|darklord|, whereas before we call it \verb|thePressureis|
	\end{exampleblock}
	
\end{frame}

%*******************************************
%*******************************************
\subsection{subfunction scheme}
\begin{frame}[fragile]
	\frametitle{\secname: \subsecname}
	
	\lstinputlisting{N3_Notes_on_programming_anonymous_functions/Pressurews.m}
	\begin{alertblock}{Main function versus sub-function}
		The first function is the main (or primary) function
		\begin{itemize}
			\item It is the only function that is accepted in the command window and other functions and script.
			\item If we run the function \verb|Pressuresubfunc| in the command window we get an error. This is because the subfunction \verb|Pressuresubfunc| is  \textbf{local} function, which is analog to the \textbf{local} variable above. 
		\end{itemize}
	\end{alertblock}
\end{frame}

%*******************************************
%*******************************************
\subsection{Input fashion}
\begin{frame}
	\frametitle{\secname: \subsecname}
	
	\lstinputlisting{N3_Notes_on_programming_anonymous_functions/Pressureinputfashion.m}
	\begin{exampleblock}{Running a function directly, notice the difference}
		Notice how we can run the function directly! this because the function does not need inputs, but the input are requested while the function is running. This is just useful to know. 
	\end{exampleblock}
\end{frame}


\begin{frame}[fragile]
	\frametitle{}
	\vspace{-3mm}
	\lstinputlisting[caption = {Filename = stats.m}] {N3_Notes_on_programming_anonymous_functions/stats.m}
	\vspace{-3mm}	
\begin{lstlisting}[caption = {Script}]
numbers       = [15 20 30 17 18]
[mean, sigma] = stats(numbers)
\end{lstlisting}
{\small Notice the assignment of the variables as outputs are different than the output names in the function (mymean vs mean). }	
\end{frame}

\subsection{More notes on the input command}

\begin{frame}
	\frametitle{\subsecname}
	
	\lstinputlisting{N3_Notes_on_programming_anonymous_functions/exercise_inputtypes.m}
	
	\begin{exampleblock}{Something useful to know}
		The input command tells the computer to wait for you to give it a value. Test the script above in your own computer. 
	\end{exampleblock}
\end{frame}




\section{Implicit Functions}
%*****************
\begin{frame}[fragile]
	
	\frametitle{Notes - 
		\href{http://www.mathworks.com/help/matlab/matlab_prog/anonymous-functions.html}{Anonymous Functions}
	}
	%\centering
	%\vspace{-17pt}

	%\begin{minipage}[t]{0.55\linewidth}
		\begin{exampleblock}{}
			\textbf{Anonymous Functions:}\\
			\begin{itemize}
				\item is a function that is not stored in a program file, but is associated with a variable.
				\item can accept inputs and return outputs, just as standard functions do. 
				\item However, they can contain only a single executable statement.
				
				\item the main of an anonymous function is that you can define in the same script so you do not have to create another file to store your function
			\end{itemize}
		\end{exampleblock}
%		\vspace{-11pt}
%		\begin{verbatim}
%		TestFun = @(x) x^3
%		%Now TestFun is a function
%		a=TestFun(3)
%		x=5
%		a=TestFun(x)
%		\end{verbatim}			
	%\end{minipage}
	\begin{alertblock}{Remember}
		Any function (implicit or explicit) is a way to outsource part of your code that would be repetitive. Outsourcing part of your code that will repeat many times into a function makes your code clear and allow you to use that same part later when you need it. 
	\end{alertblock}
	\hspace{7pt}		
	
\end{frame}

%**************************************************
%**************************************************

\subsection{Example 1}
\begin{frame}
	\frametitle{\secname: \subsecname}
	
	\lstinputlisting{N3_Notes_on_programming_anonymous_functions/anonymous_function_ex.m}
	\begin{block}{}
		Notice how we add the @() when we define the implicit function but we do not put the @ when we call the function. 
	\end{block}
\end{frame}

%**************************************************
%**************************************************

\subsection{Example 2}
\begin{frame}
	\frametitle{\secname: \subsecname}
	
	\lstinputlisting[firstline=1, lastline=16]{N3_Notes_on_programming_anonymous_functions/anonymous_function_plotting.m}
\end{frame}

\begin{frame}
	\frametitle{\secname: \subsecname\; (Continuation)}
	
	\lstinputlisting[firstline=17, lastline=47]{N3_Notes_on_programming_anonymous_functions/anonymous_function_plotting.m}
\end{frame}


%*****************
\begin{frame}[fragile]
	
	\frametitle{Notes - 
		\href{http://www.mathworks.com/help/ident/ug/algorithms-for-recursive-estimation.html}{Recursion}
	}
	%\centering
	\vspace{-17pt}
	\begin{minipage}[t]{0.47\linewidth}
		\begin{exampleblock}{}
			\textbf{Recursion:}
			\begin{itemize}
				\item is the process of defining a problem (or the solution to a problem) in terms of (a simpler version of) itself.
			\end{itemize}
		\end{exampleblock}
		%\vspace{-11pt}
		Parts of a Recursive Algorithm
		\begin{enumerate}
		\item Base Case (i.e., when to stop)
		\item Work toward Base Case
		\item Recursive Call (i.e., call ourselves)
		\end{enumerate}
		The ``work toward base case'' is where we make the problem simpler. The Base Case is the solution to the ``simplest'' possible problem. 
		\end{minipage}
	\hspace{7pt}
	\begin{minipage}[t]{0.47\linewidth}
		%\vspace{10pt}
		\begin{block}{}
			Sum up an array
		\end{block}
		\textbf{File:} \verb|SumArrayR.m|
		\begin{verbatim}
function[sum] = SumArrayR(list) 
  if (length(list) == 0 ) 
    sum = 0; 
  else 
    sum = list(1) + ...
      SumArrayR (list(2:end)); 
  end 
end 		
		\end{verbatim}			
Now try in the command line
\verb|>> list=[1:5]| \\
\verb|>> SumArrayR(list)|
	\end{minipage}
	
\end{frame}

%*****************
\begin{frame}[fragile]
	
	\frametitle{Notes - 
		\href{http://www.mathworks.com/help/ident/ug/algorithms-for-recursive-estimation.html}{Recursion}
	}
	%\centering
	\vspace{-17pt}
	\begin{minipage}[t]{0.54\linewidth}
		\begin{block}{}
		What is the definition of Factorial $!$?
		\end{block}
		\vspace{-11pt}
\[N! = N*(N-1)*(N-2)*...*2*1\]
Hmmm, but what does 
\[(N-1) * (N-2) * ... * 2*1\] equal?\\
Answer: $factorial(N-1)$

What if we combine these definitions.

Definition of Factorial:\\
\[factorial(N)=N!=N*factorial(N-1)\]
	\end{minipage}
	\hspace{7pt}
	\begin{minipage}[t]{0.41\linewidth}
		%\onslide<2->
		%\vspace{10pt}
		\begin{block}{}
			Lets write a recursive factorial function.
		\end{block}
		\textbf{File:} \verb|FindFact.m|
		\begin{verbatim}
function[res]= FindFact(N) 
   if (N == 0)  
     res = 1; 
   else 
   res = N*FindFact(N-1); 
   end
end   
\end{verbatim}			
		Now try in the command line
		\verb|>> FindFact(5)|
	\end{minipage}
\begin{exampleblock}{}
	\centering
	Compare with MATLAB's function: \href{http://www.mathworks.com/help/matlab/ref/factorial.html}{factorial(x)}
\end{exampleblock}	
\end{frame}


\section{How to continue to the next line of code}
%*****************
\begin{frame}[fragile]
	\frametitle{\secname}
	
\begin{alertblock}{This is important}
	The \textbf{ellipsis} (three consecutive periods, \verb|...|) at the end of the line means `to be continued'. This is useful to keep your code in one window. 
\end{alertblock}
\vspace{-5pt}
\begin{lstlisting}[caption = {For codes and numerical variables}]
a = [20; 21; 22]; b = {'Eng'; 'Sci';...
'Math'} %Notice the {} instead of [] 
X=table(a, b, ...
'VariableNames',{'age','Major'})
\end{lstlisting}
\vspace{-5pt}
\begin{lstlisting}[caption = {For strings}]
quote = ['the computer will do ',...
'what you tell them to do',...
'not what you want them to do']
\end{lstlisting}
\vspace{-3pt}
\begin{block}{}
\centering Test these commands/script in your computer		
\end{block}

	
\end{frame}

%*****************************************
%*****************************************

\begin{frame}
	\frametitle{How to continue your code in the same window}
	
	\lstinputlisting{N3_Notes_on_programming_anonymous_functions/conct_strings.m}
	
	\begin{block}{}
		This is useful if you want to keep your code in the same window. If you use the ellipsis MATLAB will assume the code continues in the same line, however for the programmer is easier to see all in the same window. 
	\end{block}
\end{frame}

%*****************************************
%*****************************************
\section{Vectorization}
\subsection{The most important thing to remember}
\begin{frame}[fragile]
	\frametitle{\secname: \subsecname}
	
	\begin{alertblock}{SUPER IMPORTANT}
		To make a variable into a vector can be done in two main ways. For more complex codes this might not be trivial. 
	\end{alertblock}
	%\vspace{-5pt}	
	\lstinputlisting[caption={Vectorization form 1}]{N3_Notes_on_programming_anonymous_functions/vectorization_1.m}
	%\vspace{-10pt}
	\begin{exampleblock}{}
		Notice how the variable \verb|counter| acts as a way to store the numbers into an array, while \verb|sin(vector)| is simply evaluating the sine of each value of the variable \verb|vector|
	\end{exampleblock}
\end{frame}

%*****************************************
%*****************************************
\section{Vectorization}
\subsection{The most important thing to remember}
\begin{frame}[fragile]
	\frametitle{\secname: \subsecname}
	
	\begin{alertblock}{SUPER IMPORTANT}
		To make a variable into a vector can be done in two main ways. For more complex codes this might not be trivial. 
	\end{alertblock}
	%\vspace{-5pt}	
	\lstinputlisting[caption={Vectorization form 1}]{N3_Notes_on_programming_anonymous_functions/vectorization_2.m}
	%\vspace{-10pt}
	\begin{exampleblock}{}
		Notice that on this case, we declare the variable \verb|vector| as a vector from the begining and then operate on it. This is because of MATLAB can operate directly on arrays.
	\end{exampleblock}
\end{frame}























%*****************
\placelogotrue

\begin{frame}
\frametitle{See you next class}
\vspace{-25pt}

\textbf{\textit{``Just as there is not royal road to geometry, there is no royal road to programming''}}.- Euclid and J. V. Guttag
\vspace{7pt}

\textit{The computer will do what you TELL them to do NOT what you WANT them to do}.- Someone in the Internet (Perhaps)
\vspace{7pt}	

\textit{Think twice, code once}.- Anonymous
\vspace{7pt}

\textit{The sooner you start to code, the longer the program will take}.- R. Carlson\vspace{7pt}

\textit{Any fool can write code that a computer can understand. Good programmers write code that humans can understand}.- M. Fowler
\vspace{7pt}

\textit{Simplicity is the soul of efficiency}.- A. Freeman
\vspace{7pt}

\textit{If you cannot grok the overall structure of a program while taking a shower, you are not ready to code it}.- R. Pattis

\end{frame}

\placelogofalse

\section{Appendix: Scripts included}

%\subsection{}

\begin{frame}
\frametitle{\secname}

\vspace{-7pt}
\lstinputlisting[caption={anonymous\_function\_ex.m}]{N3_Notes_on_programming_anonymous_functions/anonymous_function_ex.m}

\vspace{-7pt}
\begin{exampleblock}{}
	Try these commands in your own workstation, i.e. have the lectures on one half side of your screen and Matlab/Octave-GUI on the other half. %This is the best approach to learning this.   
\end{exampleblock}

\begin{alertblock}{}
	Check the scripts/functions under the directory for this note number (X): \newline
	/NX\_Notes\_directory
\end{alertblock}

\end{frame}	


\end{document}