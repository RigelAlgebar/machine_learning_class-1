
% Default to the notebook output style

    


% Inherit from the specified cell style.




    
\documentclass[11pt]{article}

    
    
    \usepackage[T1]{fontenc}
    % Nicer default font (+ math font) than Computer Modern for most use cases
    \usepackage{mathpazo}

    % Basic figure setup, for now with no caption control since it's done
    % automatically by Pandoc (which extracts ![](path) syntax from Markdown).
    \usepackage{graphicx}
    % We will generate all images so they have a width \maxwidth. This means
    % that they will get their normal width if they fit onto the page, but
    % are scaled down if they would overflow the margins.
    \makeatletter
    \def\maxwidth{\ifdim\Gin@nat@width>\linewidth\linewidth
    \else\Gin@nat@width\fi}
    \makeatother
    \let\Oldincludegraphics\includegraphics
    % Set max figure width to be 80% of text width, for now hardcoded.
    \renewcommand{\includegraphics}[1]{\Oldincludegraphics[width=.8\maxwidth]{#1}}
    % Ensure that by default, figures have no caption (until we provide a
    % proper Figure object with a Caption API and a way to capture that
    % in the conversion process - todo).
    \usepackage{caption}
    \DeclareCaptionLabelFormat{nolabel}{}
    \captionsetup{labelformat=nolabel}

    \usepackage{adjustbox} % Used to constrain images to a maximum size 
    \usepackage{xcolor} % Allow colors to be defined
    \usepackage{enumerate} % Needed for markdown enumerations to work
    \usepackage{geometry} % Used to adjust the document margins
    \usepackage{amsmath} % Equations
    \usepackage{amssymb} % Equations
    \usepackage{textcomp} % defines textquotesingle
    % Hack from http://tex.stackexchange.com/a/47451/13684:
    \AtBeginDocument{%
        \def\PYZsq{\textquotesingle}% Upright quotes in Pygmentized code
    }
    \usepackage{upquote} % Upright quotes for verbatim code
    \usepackage{eurosym} % defines \euro
    \usepackage[mathletters]{ucs} % Extended unicode (utf-8) support
    \usepackage[utf8x]{inputenc} % Allow utf-8 characters in the tex document
    \usepackage{fancyvrb} % verbatim replacement that allows latex
    \usepackage{grffile} % extends the file name processing of package graphics 
                         % to support a larger range 
    % The hyperref package gives us a pdf with properly built
    % internal navigation ('pdf bookmarks' for the table of contents,
    % internal cross-reference links, web links for URLs, etc.)
    \usepackage{hyperref}
    \usepackage{longtable} % longtable support required by pandoc >1.10
    \usepackage{booktabs}  % table support for pandoc > 1.12.2
    \usepackage[inline]{enumitem} % IRkernel/repr support (it uses the enumerate* environment)
    \usepackage[normalem]{ulem} % ulem is needed to support strikethroughs (\sout)
                                % normalem makes italics be italics, not underlines
    

    
    
    % Colors for the hyperref package
    \definecolor{urlcolor}{rgb}{0,.145,.698}
    \definecolor{linkcolor}{rgb}{.71,0.21,0.01}
    \definecolor{citecolor}{rgb}{.12,.54,.11}

    % ANSI colors
    \definecolor{ansi-black}{HTML}{3E424D}
    \definecolor{ansi-black-intense}{HTML}{282C36}
    \definecolor{ansi-red}{HTML}{E75C58}
    \definecolor{ansi-red-intense}{HTML}{B22B31}
    \definecolor{ansi-green}{HTML}{00A250}
    \definecolor{ansi-green-intense}{HTML}{007427}
    \definecolor{ansi-yellow}{HTML}{DDB62B}
    \definecolor{ansi-yellow-intense}{HTML}{B27D12}
    \definecolor{ansi-blue}{HTML}{208FFB}
    \definecolor{ansi-blue-intense}{HTML}{0065CA}
    \definecolor{ansi-magenta}{HTML}{D160C4}
    \definecolor{ansi-magenta-intense}{HTML}{A03196}
    \definecolor{ansi-cyan}{HTML}{60C6C8}
    \definecolor{ansi-cyan-intense}{HTML}{258F8F}
    \definecolor{ansi-white}{HTML}{C5C1B4}
    \definecolor{ansi-white-intense}{HTML}{A1A6B2}

    % commands and environments needed by pandoc snippets
    % extracted from the output of `pandoc -s`
    \providecommand{\tightlist}{%
      \setlength{\itemsep}{0pt}\setlength{\parskip}{0pt}}
    \DefineVerbatimEnvironment{Highlighting}{Verbatim}{commandchars=\\\{\}}
    % Add ',fontsize=\small' for more characters per line
    \newenvironment{Shaded}{}{}
    \newcommand{\KeywordTok}[1]{\textcolor[rgb]{0.00,0.44,0.13}{\textbf{{#1}}}}
    \newcommand{\DataTypeTok}[1]{\textcolor[rgb]{0.56,0.13,0.00}{{#1}}}
    \newcommand{\DecValTok}[1]{\textcolor[rgb]{0.25,0.63,0.44}{{#1}}}
    \newcommand{\BaseNTok}[1]{\textcolor[rgb]{0.25,0.63,0.44}{{#1}}}
    \newcommand{\FloatTok}[1]{\textcolor[rgb]{0.25,0.63,0.44}{{#1}}}
    \newcommand{\CharTok}[1]{\textcolor[rgb]{0.25,0.44,0.63}{{#1}}}
    \newcommand{\StringTok}[1]{\textcolor[rgb]{0.25,0.44,0.63}{{#1}}}
    \newcommand{\CommentTok}[1]{\textcolor[rgb]{0.38,0.63,0.69}{\textit{{#1}}}}
    \newcommand{\OtherTok}[1]{\textcolor[rgb]{0.00,0.44,0.13}{{#1}}}
    \newcommand{\AlertTok}[1]{\textcolor[rgb]{1.00,0.00,0.00}{\textbf{{#1}}}}
    \newcommand{\FunctionTok}[1]{\textcolor[rgb]{0.02,0.16,0.49}{{#1}}}
    \newcommand{\RegionMarkerTok}[1]{{#1}}
    \newcommand{\ErrorTok}[1]{\textcolor[rgb]{1.00,0.00,0.00}{\textbf{{#1}}}}
    \newcommand{\NormalTok}[1]{{#1}}
    
    % Additional commands for more recent versions of Pandoc
    \newcommand{\ConstantTok}[1]{\textcolor[rgb]{0.53,0.00,0.00}{{#1}}}
    \newcommand{\SpecialCharTok}[1]{\textcolor[rgb]{0.25,0.44,0.63}{{#1}}}
    \newcommand{\VerbatimStringTok}[1]{\textcolor[rgb]{0.25,0.44,0.63}{{#1}}}
    \newcommand{\SpecialStringTok}[1]{\textcolor[rgb]{0.73,0.40,0.53}{{#1}}}
    \newcommand{\ImportTok}[1]{{#1}}
    \newcommand{\DocumentationTok}[1]{\textcolor[rgb]{0.73,0.13,0.13}{\textit{{#1}}}}
    \newcommand{\AnnotationTok}[1]{\textcolor[rgb]{0.38,0.63,0.69}{\textbf{\textit{{#1}}}}}
    \newcommand{\CommentVarTok}[1]{\textcolor[rgb]{0.38,0.63,0.69}{\textbf{\textit{{#1}}}}}
    \newcommand{\VariableTok}[1]{\textcolor[rgb]{0.10,0.09,0.49}{{#1}}}
    \newcommand{\ControlFlowTok}[1]{\textcolor[rgb]{0.00,0.44,0.13}{\textbf{{#1}}}}
    \newcommand{\OperatorTok}[1]{\textcolor[rgb]{0.40,0.40,0.40}{{#1}}}
    \newcommand{\BuiltInTok}[1]{{#1}}
    \newcommand{\ExtensionTok}[1]{{#1}}
    \newcommand{\PreprocessorTok}[1]{\textcolor[rgb]{0.74,0.48,0.00}{{#1}}}
    \newcommand{\AttributeTok}[1]{\textcolor[rgb]{0.49,0.56,0.16}{{#1}}}
    \newcommand{\InformationTok}[1]{\textcolor[rgb]{0.38,0.63,0.69}{\textbf{\textit{{#1}}}}}
    \newcommand{\WarningTok}[1]{\textcolor[rgb]{0.38,0.63,0.69}{\textbf{\textit{{#1}}}}}
    
    
    % Define a nice break command that doesn't care if a line doesn't already
    % exist.
    \def\br{\hspace*{\fill} \\* }
    % Math Jax compatability definitions
    \def\gt{>}
    \def\lt{<}
    % Document parameters
    \title{Installing the software needed}
    
    
    

    % Pygments definitions
    
\makeatletter
\def\PY@reset{\let\PY@it=\relax \let\PY@bf=\relax%
    \let\PY@ul=\relax \let\PY@tc=\relax%
    \let\PY@bc=\relax \let\PY@ff=\relax}
\def\PY@tok#1{\csname PY@tok@#1\endcsname}
\def\PY@toks#1+{\ifx\relax#1\empty\else%
    \PY@tok{#1}\expandafter\PY@toks\fi}
\def\PY@do#1{\PY@bc{\PY@tc{\PY@ul{%
    \PY@it{\PY@bf{\PY@ff{#1}}}}}}}
\def\PY#1#2{\PY@reset\PY@toks#1+\relax+\PY@do{#2}}

\expandafter\def\csname PY@tok@w\endcsname{\def\PY@tc##1{\textcolor[rgb]{0.73,0.73,0.73}{##1}}}
\expandafter\def\csname PY@tok@c\endcsname{\let\PY@it=\textit\def\PY@tc##1{\textcolor[rgb]{0.25,0.50,0.50}{##1}}}
\expandafter\def\csname PY@tok@cp\endcsname{\def\PY@tc##1{\textcolor[rgb]{0.74,0.48,0.00}{##1}}}
\expandafter\def\csname PY@tok@k\endcsname{\let\PY@bf=\textbf\def\PY@tc##1{\textcolor[rgb]{0.00,0.50,0.00}{##1}}}
\expandafter\def\csname PY@tok@kp\endcsname{\def\PY@tc##1{\textcolor[rgb]{0.00,0.50,0.00}{##1}}}
\expandafter\def\csname PY@tok@kt\endcsname{\def\PY@tc##1{\textcolor[rgb]{0.69,0.00,0.25}{##1}}}
\expandafter\def\csname PY@tok@o\endcsname{\def\PY@tc##1{\textcolor[rgb]{0.40,0.40,0.40}{##1}}}
\expandafter\def\csname PY@tok@ow\endcsname{\let\PY@bf=\textbf\def\PY@tc##1{\textcolor[rgb]{0.67,0.13,1.00}{##1}}}
\expandafter\def\csname PY@tok@nb\endcsname{\def\PY@tc##1{\textcolor[rgb]{0.00,0.50,0.00}{##1}}}
\expandafter\def\csname PY@tok@nf\endcsname{\def\PY@tc##1{\textcolor[rgb]{0.00,0.00,1.00}{##1}}}
\expandafter\def\csname PY@tok@nc\endcsname{\let\PY@bf=\textbf\def\PY@tc##1{\textcolor[rgb]{0.00,0.00,1.00}{##1}}}
\expandafter\def\csname PY@tok@nn\endcsname{\let\PY@bf=\textbf\def\PY@tc##1{\textcolor[rgb]{0.00,0.00,1.00}{##1}}}
\expandafter\def\csname PY@tok@ne\endcsname{\let\PY@bf=\textbf\def\PY@tc##1{\textcolor[rgb]{0.82,0.25,0.23}{##1}}}
\expandafter\def\csname PY@tok@nv\endcsname{\def\PY@tc##1{\textcolor[rgb]{0.10,0.09,0.49}{##1}}}
\expandafter\def\csname PY@tok@no\endcsname{\def\PY@tc##1{\textcolor[rgb]{0.53,0.00,0.00}{##1}}}
\expandafter\def\csname PY@tok@nl\endcsname{\def\PY@tc##1{\textcolor[rgb]{0.63,0.63,0.00}{##1}}}
\expandafter\def\csname PY@tok@ni\endcsname{\let\PY@bf=\textbf\def\PY@tc##1{\textcolor[rgb]{0.60,0.60,0.60}{##1}}}
\expandafter\def\csname PY@tok@na\endcsname{\def\PY@tc##1{\textcolor[rgb]{0.49,0.56,0.16}{##1}}}
\expandafter\def\csname PY@tok@nt\endcsname{\let\PY@bf=\textbf\def\PY@tc##1{\textcolor[rgb]{0.00,0.50,0.00}{##1}}}
\expandafter\def\csname PY@tok@nd\endcsname{\def\PY@tc##1{\textcolor[rgb]{0.67,0.13,1.00}{##1}}}
\expandafter\def\csname PY@tok@s\endcsname{\def\PY@tc##1{\textcolor[rgb]{0.73,0.13,0.13}{##1}}}
\expandafter\def\csname PY@tok@sd\endcsname{\let\PY@it=\textit\def\PY@tc##1{\textcolor[rgb]{0.73,0.13,0.13}{##1}}}
\expandafter\def\csname PY@tok@si\endcsname{\let\PY@bf=\textbf\def\PY@tc##1{\textcolor[rgb]{0.73,0.40,0.53}{##1}}}
\expandafter\def\csname PY@tok@se\endcsname{\let\PY@bf=\textbf\def\PY@tc##1{\textcolor[rgb]{0.73,0.40,0.13}{##1}}}
\expandafter\def\csname PY@tok@sr\endcsname{\def\PY@tc##1{\textcolor[rgb]{0.73,0.40,0.53}{##1}}}
\expandafter\def\csname PY@tok@ss\endcsname{\def\PY@tc##1{\textcolor[rgb]{0.10,0.09,0.49}{##1}}}
\expandafter\def\csname PY@tok@sx\endcsname{\def\PY@tc##1{\textcolor[rgb]{0.00,0.50,0.00}{##1}}}
\expandafter\def\csname PY@tok@m\endcsname{\def\PY@tc##1{\textcolor[rgb]{0.40,0.40,0.40}{##1}}}
\expandafter\def\csname PY@tok@gh\endcsname{\let\PY@bf=\textbf\def\PY@tc##1{\textcolor[rgb]{0.00,0.00,0.50}{##1}}}
\expandafter\def\csname PY@tok@gu\endcsname{\let\PY@bf=\textbf\def\PY@tc##1{\textcolor[rgb]{0.50,0.00,0.50}{##1}}}
\expandafter\def\csname PY@tok@gd\endcsname{\def\PY@tc##1{\textcolor[rgb]{0.63,0.00,0.00}{##1}}}
\expandafter\def\csname PY@tok@gi\endcsname{\def\PY@tc##1{\textcolor[rgb]{0.00,0.63,0.00}{##1}}}
\expandafter\def\csname PY@tok@gr\endcsname{\def\PY@tc##1{\textcolor[rgb]{1.00,0.00,0.00}{##1}}}
\expandafter\def\csname PY@tok@ge\endcsname{\let\PY@it=\textit}
\expandafter\def\csname PY@tok@gs\endcsname{\let\PY@bf=\textbf}
\expandafter\def\csname PY@tok@gp\endcsname{\let\PY@bf=\textbf\def\PY@tc##1{\textcolor[rgb]{0.00,0.00,0.50}{##1}}}
\expandafter\def\csname PY@tok@go\endcsname{\def\PY@tc##1{\textcolor[rgb]{0.53,0.53,0.53}{##1}}}
\expandafter\def\csname PY@tok@gt\endcsname{\def\PY@tc##1{\textcolor[rgb]{0.00,0.27,0.87}{##1}}}
\expandafter\def\csname PY@tok@err\endcsname{\def\PY@bc##1{\setlength{\fboxsep}{0pt}\fcolorbox[rgb]{1.00,0.00,0.00}{1,1,1}{\strut ##1}}}
\expandafter\def\csname PY@tok@kc\endcsname{\let\PY@bf=\textbf\def\PY@tc##1{\textcolor[rgb]{0.00,0.50,0.00}{##1}}}
\expandafter\def\csname PY@tok@kd\endcsname{\let\PY@bf=\textbf\def\PY@tc##1{\textcolor[rgb]{0.00,0.50,0.00}{##1}}}
\expandafter\def\csname PY@tok@kn\endcsname{\let\PY@bf=\textbf\def\PY@tc##1{\textcolor[rgb]{0.00,0.50,0.00}{##1}}}
\expandafter\def\csname PY@tok@kr\endcsname{\let\PY@bf=\textbf\def\PY@tc##1{\textcolor[rgb]{0.00,0.50,0.00}{##1}}}
\expandafter\def\csname PY@tok@bp\endcsname{\def\PY@tc##1{\textcolor[rgb]{0.00,0.50,0.00}{##1}}}
\expandafter\def\csname PY@tok@fm\endcsname{\def\PY@tc##1{\textcolor[rgb]{0.00,0.00,1.00}{##1}}}
\expandafter\def\csname PY@tok@vc\endcsname{\def\PY@tc##1{\textcolor[rgb]{0.10,0.09,0.49}{##1}}}
\expandafter\def\csname PY@tok@vg\endcsname{\def\PY@tc##1{\textcolor[rgb]{0.10,0.09,0.49}{##1}}}
\expandafter\def\csname PY@tok@vi\endcsname{\def\PY@tc##1{\textcolor[rgb]{0.10,0.09,0.49}{##1}}}
\expandafter\def\csname PY@tok@vm\endcsname{\def\PY@tc##1{\textcolor[rgb]{0.10,0.09,0.49}{##1}}}
\expandafter\def\csname PY@tok@sa\endcsname{\def\PY@tc##1{\textcolor[rgb]{0.73,0.13,0.13}{##1}}}
\expandafter\def\csname PY@tok@sb\endcsname{\def\PY@tc##1{\textcolor[rgb]{0.73,0.13,0.13}{##1}}}
\expandafter\def\csname PY@tok@sc\endcsname{\def\PY@tc##1{\textcolor[rgb]{0.73,0.13,0.13}{##1}}}
\expandafter\def\csname PY@tok@dl\endcsname{\def\PY@tc##1{\textcolor[rgb]{0.73,0.13,0.13}{##1}}}
\expandafter\def\csname PY@tok@s2\endcsname{\def\PY@tc##1{\textcolor[rgb]{0.73,0.13,0.13}{##1}}}
\expandafter\def\csname PY@tok@sh\endcsname{\def\PY@tc##1{\textcolor[rgb]{0.73,0.13,0.13}{##1}}}
\expandafter\def\csname PY@tok@s1\endcsname{\def\PY@tc##1{\textcolor[rgb]{0.73,0.13,0.13}{##1}}}
\expandafter\def\csname PY@tok@mb\endcsname{\def\PY@tc##1{\textcolor[rgb]{0.40,0.40,0.40}{##1}}}
\expandafter\def\csname PY@tok@mf\endcsname{\def\PY@tc##1{\textcolor[rgb]{0.40,0.40,0.40}{##1}}}
\expandafter\def\csname PY@tok@mh\endcsname{\def\PY@tc##1{\textcolor[rgb]{0.40,0.40,0.40}{##1}}}
\expandafter\def\csname PY@tok@mi\endcsname{\def\PY@tc##1{\textcolor[rgb]{0.40,0.40,0.40}{##1}}}
\expandafter\def\csname PY@tok@il\endcsname{\def\PY@tc##1{\textcolor[rgb]{0.40,0.40,0.40}{##1}}}
\expandafter\def\csname PY@tok@mo\endcsname{\def\PY@tc##1{\textcolor[rgb]{0.40,0.40,0.40}{##1}}}
\expandafter\def\csname PY@tok@ch\endcsname{\let\PY@it=\textit\def\PY@tc##1{\textcolor[rgb]{0.25,0.50,0.50}{##1}}}
\expandafter\def\csname PY@tok@cm\endcsname{\let\PY@it=\textit\def\PY@tc##1{\textcolor[rgb]{0.25,0.50,0.50}{##1}}}
\expandafter\def\csname PY@tok@cpf\endcsname{\let\PY@it=\textit\def\PY@tc##1{\textcolor[rgb]{0.25,0.50,0.50}{##1}}}
\expandafter\def\csname PY@tok@c1\endcsname{\let\PY@it=\textit\def\PY@tc##1{\textcolor[rgb]{0.25,0.50,0.50}{##1}}}
\expandafter\def\csname PY@tok@cs\endcsname{\let\PY@it=\textit\def\PY@tc##1{\textcolor[rgb]{0.25,0.50,0.50}{##1}}}

\def\PYZbs{\char`\\}
\def\PYZus{\char`\_}
\def\PYZob{\char`\{}
\def\PYZcb{\char`\}}
\def\PYZca{\char`\^}
\def\PYZam{\char`\&}
\def\PYZlt{\char`\<}
\def\PYZgt{\char`\>}
\def\PYZsh{\char`\#}
\def\PYZpc{\char`\%}
\def\PYZdl{\char`\$}
\def\PYZhy{\char`\-}
\def\PYZsq{\char`\'}
\def\PYZdq{\char`\"}
\def\PYZti{\char`\~}
% for compatibility with earlier versions
\def\PYZat{@}
\def\PYZlb{[}
\def\PYZrb{]}
\makeatother


    % Exact colors from NB
    \definecolor{incolor}{rgb}{0.0, 0.0, 0.5}
    \definecolor{outcolor}{rgb}{0.545, 0.0, 0.0}



    
    % Prevent overflowing lines due to hard-to-break entities
    \sloppy 
    % Setup hyperref package
    \hypersetup{
      breaklinks=true,  % so long urls are correctly broken across lines
      colorlinks=true,
      urlcolor=urlcolor,
      linkcolor=linkcolor,
      citecolor=citecolor,
      }
    % Slightly bigger margins than the latex defaults
    
    \geometry{verbose,tmargin=1in,bmargin=1in,lmargin=1in,rmargin=1in}
    
    

    \begin{document}
    
    
    \maketitle
    
    
\tableofcontents
    
%    \section{Installing the software
%needed}\label{installing-the-software-needed}
%
%\begin{center}\rule{0.5\linewidth}{\linethickness}\end{center}
%
%\subsection{Table of Contents}\label{table-of-contents}
%
%\begin{enumerate}
%\def\labelenumi{\arabic{enumi}.}
%\tightlist
%\item
%  Section \ref{install_matlab}
%\item
%  Section \ref{install_octave} \textgreater{} 1.
%  Section \ref{install_octave_windows}
%\item
%  Section \ref{install_octave_linux}
%\item
%  Section \ref{install_octave_mac_10}
%\item
%  Section \ref{install_octave_mac_earlier}
%\item
%  Section \ref{install_python}
%\item
%  Section \ref{tutorials_matlab}
%\item
%  Section \ref{tutorials_octave}
%\item
%  Section \ref{tutorials_python} ***
%\end{enumerate}

\section{Installing MATLAB}\label{installing-matlab}

MathWorks is providing you access to MATLAB for use in your coursework.
When planning your activity, please note that access is valid for the
length of the course (12 weeks).

\begin{enumerate}
\def\labelenumi{\arabic{enumi}.}
\item
  Enter your email to create a MathWorks account if you do not have one.
  https://www.mathworks.com/downloads/web\_downloads/
\item
  Use this
  \textbf{\href{https://www.mathworks.com/downloads/web_downloads/}{link}}
  again to download and install. You may need to log-in to your
  MathWorks account that you created in Step 1. After starting the
  installer, accept all defaults and log-in to your MathWorks account
  when prompted.
\end{enumerate}

\textbf{Note:} If you can not download it, let me know and I can help
you.

For additional resources, including an introduction to the MATLAB
interface, please see "More Octave/MATLAB Resources."

\section{Installing Octave (GUI)}\label{installing-octave-gui}

\subsection{Installing Octave on
Windows}\label{installing-octave-on-windows}

Use this link to install Octave for windows:
https://www.gnu.org/software/octave/ and
http://wiki.octave.org/Octave\_for\_Microsoft\_Windows

Octave on Windows can be used to submit programming assignments in this
course but will likely need a patch provided in the discussion forum.
Refer to https://www.gnu.org/software/octave/ for more information about
the patch for your version.

"Warning: Do not install Octave 4.0.0"; checkout the "Resources" menu's
section of "Installation Issues".

\subsection{Installing Octave on
GNU/Linux}\label{installing-octave-on-gnulinux}

We recommend
\textbf{\href{http://wiki.octave.org/Octave_for_GNU/Linux}{using your
system package manager to install Octave}}.

On Ubuntu, you can use: -
\texttt{sudo\ apt-get\ update\ \&\&\ sudo\ apt-get\ install\ octave}

On Fedora, you can use: - \texttt{sudo\ yum\ install\ octave-forge}

Please consult
\textbf{\href{http://wiki.octave.org/Octave_for_GNU/Linux}{the Octave
maintainer's instructions}} for other GNU/Linux systems.

"Warning: Do not install Octave 4.0.0"; checkout the "Resources" menu's
section of "Installation Issues".

\subsection{Installing Octave on Mac OS X (10.10 Yosemite and 10.9
Mavericks)}\label{installing-octave-on-mac-os-x-10.10-yosemite-and-10.9-mavericks}

\begin{enumerate}
\def\labelenumi{\arabic{enumi}.}
\item
  Mac OS X \textbf{\href{https://support.apple.com/en-us/HT202491}{has a
  feature called Gatekeeper}} has a feature called Gatekeeper that may
  only let you install applications from the Mac App Store. You may need
  to configure it to allow the Octave installer. Visit your System
  Preferences, click Security \& Privacy, and check the setting to allow
  apps downloaded from Anywhere. You may need to enter your password to
  unlock the settings page.
\item
  Download \textbf{\href{https://wiki.octave.org/Octave_for_macOS}{the
  Octave 3.8.0 installer}} or the latest version that isn't 4.0.0. The
  file is large so this may take some time.
\item
  Open the downloaded image, probably named GNU\_Octave\_3.8.0-6.dmg on
  your computer, and then open Octave-3.8.0-6.mpkg inside.
\item
  Follow the installer's instructions. You may need to enter the
  administrator password for your computer.
\item
  After the installer completes, Octave should be installed on your
  computer. You can find Octave-cli in your Mac's Applications, which is
  a text interface for Octave that you can use to complete Machine
  Learning's programming assignments.
\end{enumerate}

Octave also includes an experimental graphical interface which is called
Octave-gui, also in your Mac's Applications, but we recommend using
Octave-cli because it's more stable.

Note: If you use a package manager (like MacPorts or Homebrew), we
recommend you follow
\textbf{\href{http://wiki.octave.org/Octave_for_macOS\#Package_Managers}{the
package manager installation instructions}}.

"Warning: Do not install Octave 4.0.0"; checkout the "Resources" menu's
section of "Installation Issues".

\subsection{Installing Octave on Mac OS X (10.8 Mountain Lion and
Earlier)}\label{installing-octave-on-mac-os-x-10.8-mountain-lion-and-earlier}

Installing Octave on Mac OS X (10.8 Mountain Lion and Earlier) If you
use Mac OS X 10.9, we recommend following the instructions
\textbf{Section \ref{install_octave_mac_10}}. For other Mac OS X
versions, the Octave project doesn't distribute installers. We recommend
installing Homebrew, a package manager, using
\textbf{\href{http://wiki.octave.org/Octave_for_macOS\#Homebrew}{their
instructions}}.

"Warning: Do not install Octave 4.0.0"; checkout the "Resources" menu's
section of "Installation Issues".

\section{Installing Python 3}\label{installing-python-3}

I recommend to install
\textbf{\href{https://www.anaconda.com/download/}{Anaconda}}. If you do
need to install Python and aren't confident about the task you can find
a few notes on the
\textbf{\href{https://wiki.python.org/moin/BeginnersGuide/Download}{BeginnersGuide/Download
wiki}} page, but installation is unremarkable on most platforms.

    \section{Tutorials for Matlab}\label{tutorials-for-matlab}

At the MATLAB command line, typing help followed by a function name
displays documentation for a built-in function. For example, help plot
will bring up help information for plotting. Further documentation can
be found at the MATLAB
\textbf{\href{http://www.mathworks.com/help/matlab/}{documentation
pages}}.

\subsection{Introduction to MATLAB}\label{introduction-to-matlab}

MathWorks also has a series of videos about various MATLAB features:

\begin{longtable}[]{@{}ll@{}}
\toprule
\begin{minipage}[b]{0.29\columnwidth}\raggedright\strut
\textbf{Learning Module}\strut
\end{minipage} & \begin{minipage}[b]{0.27\columnwidth}\raggedright\strut
\textbf{Learning Goals}\strut
\end{minipage}\tabularnewline
\midrule
\endhead
\begin{minipage}[t]{0.29\columnwidth}\raggedright\strut
\textbf{\href{http://youtu.be/rXwTiKGlilE}{What is MATLAB?}}\strut
\end{minipage} & \begin{minipage}[t]{0.27\columnwidth}\raggedright\strut
Introduce MATLAB\strut
\end{minipage}\tabularnewline
\begin{minipage}[t]{0.29\columnwidth}\raggedright\strut
\textbf{\href{http://youtu.be/iYTzJXXI9vI}{The MATLAB
Environment}}\strut
\end{minipage} & \begin{minipage}[t]{0.27\columnwidth}\raggedright\strut
Navigate the command line, workspace, directory, and editor\strut
\end{minipage}\tabularnewline
\begin{minipage}[t]{0.29\columnwidth}\raggedright\strut
\textbf{\href{http://youtu.be/jURDBsIPt5I}{MATLAB Variables}}\strut
\end{minipage} & \begin{minipage}[t]{0.27\columnwidth}\raggedright\strut
Use the assignment operator to define scalar variables\strut
\end{minipage}\tabularnewline
\begin{minipage}[t]{0.29\columnwidth}\raggedright\strut
\textbf{\href{http://youtu.be/E7KllorEWkA}{MATLAB as a
Calculator}}\strut
\end{minipage} & \begin{minipage}[t]{0.27\columnwidth}\raggedright\strut
Perform arithmetic calculations with scalars and functions using MATLAB
syntax and order of operations.\strut
\end{minipage}\tabularnewline
\begin{minipage}[t]{0.29\columnwidth}\raggedright\strut
\textbf{\href{http://youtu.be/R-kBvJ3kVVk}{Mathematical
Functions}}\strut
\end{minipage} & \begin{minipage}[t]{0.27\columnwidth}\raggedright\strut
Use MATLAB variables for input and output to functions. Examples
include: COS, SIN, EXP, and NTHROOT.\strut
\end{minipage}\tabularnewline
\bottomrule
\end{longtable}

\subsection{Vectors}\label{vectors}

\begin{longtable}[]{@{}ll@{}}
\toprule
\begin{minipage}[b]{0.29\columnwidth}\raggedright\strut
\textbf{Learning Module}\strut
\end{minipage} & \begin{minipage}[b]{0.29\columnwidth}\raggedright\strut
\textbf{Learning Goals}\strut
\end{minipage}\tabularnewline
\midrule
\endhead
\begin{minipage}[t]{0.29\columnwidth}\raggedright\strut
\textbf{\href{http://youtu.be/2VNFqxmVqw8}{Creating Vectors via
Concatenation}}\strut
\end{minipage} & \begin{minipage}[t]{0.29\columnwidth}\raggedright\strut
Create vectors by entering individual elements\strut
\end{minipage}\tabularnewline
\begin{minipage}[t]{0.29\columnwidth}\raggedright\strut
\textbf{\href{http://youtu.be/GihLWwp8sBw}{Accessing Elements of a
Vector}}\strut
\end{minipage} & \begin{minipage}[t]{0.29\columnwidth}\raggedright\strut
Access specific elements of a vector\strut
\end{minipage}\tabularnewline
\begin{minipage}[t]{0.29\columnwidth}\raggedright\strut
\textbf{\href{http://youtu.be/t9Kla_YFdfs}{Vector Arithmetic}}\strut
\end{minipage} & \begin{minipage}[t]{0.29\columnwidth}\raggedright\strut
Perform arithmetic calculations with vectors including element-wise
operations\strut
\end{minipage}\tabularnewline
\begin{minipage}[t]{0.29\columnwidth}\raggedright\strut
\textbf{\href{http://youtu.be/USehPX2iEa4}{Vector Transpose}}\strut
\end{minipage} & \begin{minipage}[t]{0.29\columnwidth}\raggedright\strut
Use the transpose operator to convert between row and column
vectors\strut
\end{minipage}\tabularnewline
\begin{minipage}[t]{0.29\columnwidth}\raggedright\strut
\textbf{\href{http://youtu.be/L7cERR5J9XY}{Creating Uniformly Spaced
Vectors (The Colon Operator)}}\strut
\end{minipage} & \begin{minipage}[t]{0.29\columnwidth}\raggedright\strut
Use the colon operator syntax to create vectors given the starting and
ending values and the size of the interval\strut
\end{minipage}\tabularnewline
\begin{minipage}[t]{0.29\columnwidth}\raggedright\strut
\textbf{\href{http://youtu.be/3QM3LRnb4Tw}{Creating Uniformly Spaced
Vectors (The LINSPACE Function)}}\strut
\end{minipage} & \begin{minipage}[t]{0.29\columnwidth}\raggedright\strut
Use the LINSPACE function to create a vector.\strut
\end{minipage}\tabularnewline
\bottomrule
\end{longtable}

\subsection{Visualization}\label{visualization}

\begin{longtable}[]{@{}ll@{}}
\toprule
\begin{minipage}[b]{0.29\columnwidth}\raggedright\strut
** Learning Module **\strut
\end{minipage} & \begin{minipage}[b]{0.29\columnwidth}\raggedright\strut
\textbf{Learning Goals}\strut
\end{minipage}\tabularnewline
\midrule
\endhead
\begin{minipage}[t]{0.29\columnwidth}\raggedright\strut
\textbf{\href{http://youtu.be/00k9A9W0cl8}{Line Plots}}\strut
\end{minipage} & \begin{minipage}[t]{0.29\columnwidth}\raggedright\strut
Create a line plot of a vector and customize plot markers and
colors\strut
\end{minipage}\tabularnewline
\begin{minipage}[t]{0.29\columnwidth}\raggedright\strut
\textbf{\href{http://youtu.be/ab3XIDdloNI}{Annotating Graphs}}\strut
\end{minipage} & \begin{minipage}[t]{0.29\columnwidth}\raggedright\strut
Label axes, add a title, and add a legend to a plot\strut
\end{minipage}\tabularnewline
\bottomrule
\end{longtable}

\subsection{Matrices and Arrays}\label{matrices-and-arrays}

\begin{longtable}[]{@{}ll@{}}
\toprule
\begin{minipage}[b]{0.29\columnwidth}\raggedright\strut
\textbf{Learning Module}\strut
\end{minipage} & \begin{minipage}[b]{0.29\columnwidth}\raggedright\strut
\textbf{Learning Goals}\strut
\end{minipage}\tabularnewline
\midrule
\endhead
\begin{minipage}[t]{0.29\columnwidth}\raggedright\strut
\textbf{\href{http://youtu.be/5tm6PKaJdI8}{Creating Matrices}}\strut
\end{minipage} & \begin{minipage}[t]{0.29\columnwidth}\raggedright\strut
Create matrices by directly entering scalars\strut
\end{minipage}\tabularnewline
\begin{minipage}[t]{0.29\columnwidth}\raggedright\strut
\textbf{\href{http://youtu.be/DDnm7vek6KY}{Array Creation
Functions}}\strut
\end{minipage} & \begin{minipage}[t]{0.29\columnwidth}\raggedright\strut
Create larger matrices and vectors with built in MATLAB functions such
as ZEROS and EYE\strut
\end{minipage}\tabularnewline
\begin{minipage}[t]{0.29\columnwidth}\raggedright\strut
\textbf{\href{http://youtu.be/qqQnFp5aiuM}{Accessing Elements of an
Array}}\strut
\end{minipage} & \begin{minipage}[t]{0.29\columnwidth}\raggedright\strut
Access elements of an array including entire columns or rows using
row-column indexing.\strut
\end{minipage}\tabularnewline
\begin{minipage}[t]{0.29\columnwidth}\raggedright\strut
\textbf{\href{http://youtu.be/SqvtT_VspKU}{Array Size and Length}}\strut
\end{minipage} & \begin{minipage}[t]{0.29\columnwidth}\raggedright\strut
Use built-in functions to determine array dimensions\strut
\end{minipage}\tabularnewline
\begin{minipage}[t]{0.29\columnwidth}\raggedright\strut
\textbf{\href{http://youtu.be/TgopxS-_zl8}{Concatenating Arrays}}\strut
\end{minipage} & \begin{minipage}[t]{0.29\columnwidth}\raggedright\strut
Build larger arrays from smaller ones\strut
\end{minipage}\tabularnewline
\begin{minipage}[t]{0.29\columnwidth}\raggedright\strut
\textbf{\href{http://youtu.be/-jgXqAYBhxI}{Matrix Multiplication}}\strut
\end{minipage} & \begin{minipage}[t]{0.29\columnwidth}\raggedright\strut
Perform matrix multiplication and interpret error messages related to
incompatible dimensions.\strut
\end{minipage}\tabularnewline
\bottomrule
\end{longtable}

\subsection{Programming}\label{programming}

\begin{longtable}[]{@{}ll@{}}
\toprule
\begin{minipage}[b]{0.29\columnwidth}\raggedright\strut
\textbf{Learning Module}\strut
\end{minipage} & \begin{minipage}[b]{0.29\columnwidth}\raggedright\strut
\textbf{Learning Goals}\strut
\end{minipage}\tabularnewline
\midrule
\endhead
\begin{minipage}[t]{0.29\columnwidth}\raggedright\strut
\textbf{\href{http://youtu.be/TZr6GyxnI_w}{Using the MATLAB
Editor}}\strut
\end{minipage} & \begin{minipage}[t]{0.29\columnwidth}\raggedright\strut
Write a script in the MATLAB Editor, break code into sections to
execute, and find help on functions\strut
\end{minipage}\tabularnewline
\begin{minipage}[t]{0.29\columnwidth}\raggedright\strut
\textbf{\href{http://youtu.be/5gVKJVVmbrM}{Logical Operators}}\strut
\end{minipage} & \begin{minipage}[t]{0.29\columnwidth}\raggedright\strut
Use relational and logical operators to create logical variables for
program control\strut
\end{minipage}\tabularnewline
\begin{minipage}[t]{0.29\columnwidth}\raggedright\strut
\textbf{\href{http://youtu.be/8wxh4LtT--g}{Conditional Data
Selection}}\strut
\end{minipage} & \begin{minipage}[t]{0.29\columnwidth}\raggedright\strut
Access and change elements for a vector the meet a specified
criteria\strut
\end{minipage}\tabularnewline
\begin{minipage}[t]{0.29\columnwidth}\raggedright\strut
\textbf{\href{http://youtu.be/oaK2-ZT9dls}{If-Else Statements}}\strut
\end{minipage} & \begin{minipage}[t]{0.29\columnwidth}\raggedright\strut
Use if-else statements to control which lines of code are
evaluated\strut
\end{minipage}\tabularnewline
\begin{minipage}[t]{0.29\columnwidth}\raggedright\strut
\textbf{\href{http://youtu.be/1u3RahlWEZA}{For Loops}}\strut
\end{minipage} & \begin{minipage}[t]{0.29\columnwidth}\raggedright\strut
Repeat a sequence of commands a specified number of times\strut
\end{minipage}\tabularnewline
\begin{minipage}[t]{0.29\columnwidth}\raggedright\strut
\textbf{\href{http://youtu.be/dofj51Ovdl4}{While Loops}}\strut
\end{minipage} & \begin{minipage}[t]{0.29\columnwidth}\raggedright\strut
Repeat a sequence of commands while a specified condition is true\strut
\end{minipage}\tabularnewline
\bottomrule
\end{longtable}

\subsection{Learn to Code in Matlab}\label{learn-to-code-in-matlab}

 https://learntocode.mathworks.com/portal.html

This is a very interactive course with videos, exercises, made by the
creators of Matlab. This will help you with the fundamentals and steps
by step process for most of what is needed for the course. It is written
for a wide audience so the steps are extremely clear, with simple
language and each lesson have a video with exercises. You do not need an
account to take the tutorial.

Other sets of videos targetting machine learning building blocks are
described below.

\section{Tutorials for Octave}\label{tutorials-for-octave}

At the Octave command line, typing \texttt{help} followed by a function
name displays documentation for a built-in function. For example,
\texttt{help\ plot} will bring up help information for plotting. Further
documentation can be found at the Octave
\textbf{\href{http://www.gnu.org/software/octave/doc/interpreter/}{documentation
pages}}.

\section{Tutorials for Python}\label{tutorials-for-python}

At the Python console, typing help({[}object{]}) invokes the built-in
help system. If no argument is given, the interactive help system starts
on the interpreter console. If the argument is a string, then the string
is looked up as the name of a module, function, class, method, keyword,
or documentation topic, and a help page is printed on the console. If
the argument is any other kind of object, a help page on the object is
generated. Further documentation can be found at
\textbf{\href{https://docs.python.org/3/}{documentation for python
3.6.4.}}.

If you've never programmed before, the tutorials on this
\textbf{\href{https://wiki.python.org/moin/BeginnersGuide/NonProgrammers}{beginners
guide}} are recommended for you; they don't assume that you have
previous experience. When you're learning, small
\textbf{\href{https://wiki.python.org/moin/BeginnersGuide/Examples}{examples}}
can be very helpful. If you have programming experience, also check out
the
\textbf{\href{https://wiki.python.org/moin/BeginnersGuide/Programmers}{beginnersGuide/programmers}}
page and the
\textbf{\href{https://docs.python.org/3/py-modindex.html}{python module
index}}.

\subsection{Programming in Python}\label{programming-in-python}

\begin{longtable}[]{@{}ll@{}}
\toprule
\begin{minipage}[b]{0.29\columnwidth}\raggedright\strut
\textbf{Learning Module}\strut
\end{minipage} & \begin{minipage}[b]{0.29\columnwidth}\raggedright\strut
\textbf{Learning Goals}\strut
\end{minipage}\tabularnewline
\midrule
\endhead
\begin{minipage}[t]{0.29\columnwidth}\raggedright\strut
\textbf{\href{http://interactivepython.org/runestone/static/thinkcspy/PythonTurtle/TheforLoop.html}{For
loop}}\strut
\end{minipage} & \begin{minipage}[t]{0.29\columnwidth}\raggedright\strut
Repeat a sequence of commands a specified number of times\strut
\end{minipage}\tabularnewline
\begin{minipage}[t]{0.29\columnwidth}\raggedright\strut
\textbf{\href{http://interactivepython.org/runestone/static/thinkcspy/MoreAboutIteration/ThewhileStatement.html}{While}}\strut
\end{minipage} & \begin{minipage}[t]{0.29\columnwidth}\raggedright\strut
Repeat a sequence of commands as far as a condition is satisfied\strut
\end{minipage}\tabularnewline
\begin{minipage}[t]{0.29\columnwidth}\raggedright\strut
\textbf{\href{http://interactivepython.org/runestone/static/thinkcspy/Functions/toctree.html}{Functions}}\strut
\end{minipage} & \begin{minipage}[t]{0.29\columnwidth}\raggedright\strut
Use Python variables for input and output to functions.\strut
\end{minipage}\tabularnewline
\begin{minipage}[t]{0.29\columnwidth}\raggedright\strut
\textbf{\href{http://interactivepython.org/runestone/static/thinkcspy/Strings/toctree.html}{Strings}}\strut
\end{minipage} & \begin{minipage}[t]{0.29\columnwidth}\raggedright\strut
Manipulation of characters\strut
\end{minipage}\tabularnewline
\begin{minipage}[t]{0.29\columnwidth}\raggedright\strut
\textbf{\href{http://interactivepython.org/runestone/static/thinkcspy/Lists/toctree.html}{List}}\strut
\end{minipage} & \begin{minipage}[t]{0.29\columnwidth}\raggedright\strut
Sequential collection of Python data values, where each value is
identified by an index\strut
\end{minipage}\tabularnewline
\begin{minipage}[t]{0.29\columnwidth}\raggedright\strut
\textbf{\href{http://interactivepython.org/runestone/static/thinkcspy/Files/toctree.html}{Files}}\strut
\end{minipage} & \begin{minipage}[t]{0.29\columnwidth}\raggedright\strut
How to work with Files\strut
\end{minipage}\tabularnewline
\begin{minipage}[t]{0.29\columnwidth}\raggedright\strut
\textbf{\href{http://interactivepython.org/runestone/static/thinkcspy/Dictionaries/toctree.html}{Dictionaries}}\strut
\end{minipage} & \begin{minipage}[t]{0.29\columnwidth}\raggedright\strut
Python's built-in mapping type.\strut
\end{minipage}\tabularnewline
\bottomrule
\end{longtable}


    % Add a bibliography block to the postdoc
    
    
    
    \end{document}
