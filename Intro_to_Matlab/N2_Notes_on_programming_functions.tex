\documentclass[11pt]{beamer}

%START To get MATLAB environment
\usepackage[numbered,framed]{matlab-prettifier}
%\usepackage{filecontents}

\let\ph\mlplaceholder % shorter macro
\lstMakeShortInline"

\lstset{
	style              = Matlab-editor,
	basicstyle         = \tiny \ttfamily,
	escapechar         = ",
	mlshowsectionrules = true,
}

%\renewcommand{\lstlistingname}{Algorithm}% Listing -> Algorithm
\renewcommand{\lstlistingname}{Code}% Listing -> Code
%FINISH To get MATLAB environment

%\usepackage{standalone}
%\graphicspath{{../}{../}}
%\graphicspath{{../../figures/}}

\setcounter{tocdepth}{1}

\usepackage[english]{babel}

\usepackage{amsmath}
\usepackage{amsfonts}
\usepackage{amssymb}
\usepackage{graphicx}
\usepackage{tikz}
\usetikzlibrary{shapes.geometric, arrows}
\tikzstyle{startstop} = [rectangle, rounded corners, minimum width=1.7cm, minimum height=0.7cm,text centered, draw=black, fill=red!30]
\tikzstyle{io} = [trapezium, trapezium left angle=70, trapezium right angle=110, minimum width=1.7cm, minimum height=0.7cm, text centered, draw=black, fill=blue!30]
\tikzstyle{process} = [rectangle, minimum width=1.7cm, minimum height=0.7cm, text centered, draw=black, fill=orange!30]
\tikzstyle{decision} = [diamond, minimum width=1.7cm, minimum height=0.7cm, text centered, text width=1.7cm, draw=black, fill=green!30]
\tikzstyle{arrow} = [thick,->,>=stealth]



\setbeamercovered{transparent}
%\usepackage{enumitem}
%\setlist[itemize]{leftmargin=*}

%\usepackage{mhchem}
%\usepackage[utf8]{inputenc}
%\usepackage[T1]{fontenc}
%\numberwithin{equation}{section}



\author[Jose Mendoza-Cortes]{Prof. Jose L. Mendoza-Cortes}
\title[Machine Learning]{Machine Learning}
%\subtitle{Spring '18}
\institute[]
{\scriptsize  
	Scientific Computing Department, Dirac Science Building \\
	Materials Science and Engineering, High Performance Materials Institute\\
	Florida State University\\
	\href{mailto:jmendozacortes@fsu.edu}{jmendozacortes@fsu.edu}\\[3mm]
	
	Condensed Matter Theory, National High Magnetic Field Laboratory\\%[3mm]
	Florida State University\\	
	\href{mailto:mendoza@magnet.fsu.edu}{mendoza@magnet.fsu.edu}\\[3mm]	
	
	Chemical and Biomedical Engineering \\
	Florida State University | Florida A\&M University | College of Engineering \\
	\href{mailto:mendoza@eng.famu.fsu.edu}{mendoza@eng.famu.fsu.edu}\\[3mm]
	Web: \href{http://mendoza.eng.fsu.edu/}{http://mendoza.eng.fsu.edu/}\\%[1mm]
}  

\date{}
\subject{Theory and Computations in Materials, Chemistry and Physics}

%\usetheme{Berkeley}
%\logo{\includegraphics[scale=0.213]{figures/fsu_logo.png}}
%\addtobeamertemplate{navigation symbols}{}{%
%    \usebeamerfont{footline}%
%    \usebeamercolor[fg]{footline}%
%    \hspace{1em}%
%    \insertframenumber/\inserttotalframenumber
%}

\usetheme{Madrid}
%\usecolortheme{beaver}
%\usecolortheme{orchid}

\newif\ifplacelogo % create a new conditional
\placelogotrue % set it to true
\logo{\ifplacelogo
	\includegraphics[width=0.1\linewidth]{figures/fsu_logo.png}
	\includegraphics[width=0.1\linewidth]{figures/famu_logo.png}
	\includegraphics[width=0.1\linewidth]{figures/maglab_logo.png}
	\fi} % replace with your own command

\definecolor{mycustom}{RGB}{0,0,102}       %102,38,38 %128,0,0
%\definecolor{mycustom}{RGB}{128,0,0}       %102,38,38 %128,0,0
%
%\setbeamercolor{structure}{bg=white, fg=custom}
%\setbeamercolor{caption}{fg=custom}

\definecolor{custom}{cmyk}{1,0.5,0,0.47}       %102,38,38 %128,0,0

\setbeamercolor{structure}{bg=white, fg=custom}
\setbeamercolor{caption}{fg=custom}

\setbeamertemplate{navigation symbols}{} %To remove the navigation symbols

%\setbeamercolor{frametitle}{fg=custom}
%\setbeamercolor{framesubtitle}{fg=custom}
\setbeamercolor{titlelike}{parent=structure,bg=gray!20!white}

\setlength\abovecaptionskip{-3pt}
\setbeamertemplate{caption}{%
	\insertcaptionname\,\insertcaptionnumber:\,\insertcaption
}



\usepackage{hyperref}
\hypersetup{colorlinks=true,
	linkcolor=mycustom,
	urlcolor=mycustom}

\abovedisplayskip=0pt
\belowdisplayskip=0pt


	\usepackage{pgfpages}
	\pgfpagesuselayout{2 on 1}[letterpaper,%landscape,
	border shrink=5mm]


\begin{document}

\placelogotrue % turn the logo off \usetheme{Madrid}
\maketitle

\placelogofalse % turn the logo off


\section{Notes}
%*****************
\section{\href{http://www.mathworks.com/help/matlab/ref/function.html}{Functions} and \href{http://www.mathworks.com/help/matlab/learn_matlab/scripts-and-functions.html}{Scripts}}

\subsection{Introduction}

\begin{frame}[fragile]
	\frametitle{\secname : \subsecname}
	%\centering
	
	\begin{block}{}
		\centering
		Notes: Functions and Visualization \\
	\end{block}
	\begin{minipage}[t]{0.43\linewidth}
	\vspace{-10pt}
		\begin{enumerate}
			\item Review last class 
			\item Functions 
			\item Visualization
			\item Combination with flow diagram
		\end{enumerate}
		\vspace{-10pt}
		\begin{alertblock}{Remember}
			\begin{itemize}
			\item The percent sign \verb|%| denotes
			the start of a comment, and MATLAB ignores it.\\
			\item The operators \verb|.*| and \verb|./| tell MATLAB to do element by element multiplication
			\end{itemize}
		\end{alertblock}
	\end{minipage}
	\hspace{10pt}
	\begin{minipage}[t]{0.51\linewidth}
		\vspace{-15pt}
		\begin{block}{How to look for Help}
			\begin{itemize}
				\item You can always get help on a command (say \verb|plot|) by typing \verb|help plot| in MATLAB's command window.
				\item You can also use the upper right corner section called ``Search Documentation''
				\item And of course, there is also \href{www.google.com}{Google}. Just make sure that in your search you include 'MATLAB and the question'
			\end{itemize}			
		\end{block}
	\end{minipage}
\end{frame}


%*****************
\begin{frame}[fragile]
	
	\frametitle{\secname : \subsecname}
		%\centering
    \vspace{-5pt}
	\begin{block}{}
		It is far more powerful, and liberating to use it in a ``batch'' mode using:
	\end{block}
    \vspace{-15pt}
	\begin{minipage}[t]{0.47\linewidth}
		\begin{exampleblock}{Functions}
		provide definitions of new functions
		\end{exampleblock}
		%\vspace{-10pt}
		Example file \verb|Pressure.m|
		\begin{verbatim}
		function P = Pressure(T,V,R)
		% Pressure(a,b,c) returns 
		% ideal gas pressure
		P = R.*T./V; 
		end
		\end{verbatim}
		%\vspace{-5pt}
		You can use this function from the command window as:
		\vspace{-10pt}		
		\begin{verbatim}
		Pressure(298,0.022,8.314)
		\end{verbatim}		
	\end{minipage}
	\hspace{10pt}
	\begin{minipage}[t]{0.47\linewidth}
		\begin{block}{scripts}
		list of commands to be executed.
		\end{block}
		Example file \verb|IdealGasScript.m|
		\begin{verbatim}
		% script M-file IdealGasScript.m
		R = 8.314; % Gas Constant
		T = 298;
		V = 0.0022;
		Pressure(T,V,R)
		\end{verbatim}
		You can run this file from the \verb|Run Script| menu or by typing the
		filename without the .m on the command line. 
	\end{minipage}
	
\end{frame}


%*****************
\begin{frame}[fragile]
	
	\frametitle{\secname : \subsecname}
	%\centering
	\vspace{-5pt}
	\begin{alertblock}{Be careful}
		\begin{itemize}
			\item[] \textbf{Functions}
			\item When you call the function \verb|Pressure| MATLAB looks for the file \verb|Pressure.m| 
			in the current directory and in the MATLAB \href{http://www.mathworks.com/help/matlab/matlab_env/add-remove-or-reorder-folders-on-the-search-path.html}{search path}. 
			\item If you want to change a function or a script you must tell MATLAB to
			use the new version by entering the command \verb|clear Pressure| from the
			command line to remove the old version from memory. The command \verb|clear all| 
			removes all definitions so can start with a blank slate of MATLAB. Use this command
			with great care. 
			\item[] \textbf{scripts}
			\item The script file \verb|IdealGasScript.m| must be in your MATLAB \href{http://www.mathworks.com/help/matlab/matlab_env/add-remove-or-reorder-folders-on-the-search-path.html}{path}
			\item If script files are not doing what you expect the command \verb|clear all| will remove all old
			definitions from memory. This is often necessary when you are changing the script file. Again, ``with great care, you should use this command''
		\end{itemize}
	\end{alertblock}
	\vspace{-15pt}
	\begin{minipage}[t]{0.47\linewidth}

	\end{minipage}
	\hspace{10pt}
	\begin{minipage}[t]{0.47\linewidth}
		
	\end{minipage}
	
\end{frame}

\subsection{Definition}
%*****************
\begin{frame}[fragile]
	
	\frametitle{\secname : \subsecname}
	%\centering
	\vspace{-1pt}
	\begin{exampleblock}{Functions (according to MATLAB)}
		\begin{itemize}
			\item[] \textbf{Syntax}			
			\item[] \verb|function [y1,...,yN] = funcname(x1,...,xM)|
			\item A function files is defined by writing \verb|function| in the first line
			\item That accepts inputs \verb|x1,...,xM| \phantom{123}\%notice the (~)
			\item Return outputs \verb|y1,...,yN| \phantom{1235678}\%notice the [~]
			\item The name of the file should match the name of the first function in the file (e.g. \verb|funcname.m|). Valid function names begin with an alphabetic character, and can contain letters, numbers, or underscores.			
		\end{itemize}
	\end{exampleblock}
	\vspace{-2mm}
	\begin{center}
	\includegraphics[width=0.7\linewidth]{figures/function}
	\end{center}
	
\end{frame}

%*****************
\begin{frame}[fragile]
	
	\frametitle{\secname : \subsecname}
	%\centering
	\vspace{-11pt}
	\begin{exampleblock}{Functions (Helpcomments definition)}
		\begin{itemize}
			\item[] \textbf{Syntax}			
			\item[] \verb|function [y1,...,yN] = funcname(x1,...,xM)|
			\item[] \verb|%HELPCOMMENTS (1st line): Description of the function|
			\item[] \verb|%HELPCOMMENTS (2nd line): Use of the function (How to)|						
			\item[] \verb|%HELPCOMMENTS (3rd line): Definition of Inputs|
			\item[] \verb|%HELPCOMMENTS (4th line): Definition of Outputs|			
			\end{itemize}
	\end{exampleblock}
	\vspace{0mm}
\begin{alertblock}{Helpcomments}
		\begin{itemize}
			\item The HELPCOMMENTS section is a very important part of creating a function, perhaps the most important
			\item 1st line of HELPCOMMENTS will be found by the command \verb|>>lookfor word| where word is in the 1st line of your function
			\item All the lines from 1st - $n$ line in Helpcomments will be found by the command \verb|>>Help funcname|			
		\end{itemize}
\end{alertblock}
	
\end{frame}


\subsection{Examples}
%*****************
%*****************
\begin{frame}[fragile]
	
	\frametitle{\secname : \subsecname}
	%\centering
	%\vspace{-29pt}
%	\begin{block}{Functions (more things to remember)}
%\textbf{Example where all the Helpcomments are being used.}
\vspace{-5pt}
\textbf{File:} \verb|Absolute.m|
\begin{exampleblock}{Using the HELPCOMMENTS}
\begin{verbatim}
function [ numberabs ] = Absolute( anynumber )
%Description:This func takes the absolute value of a number
%Use:        numberabs = Absolute (anynumber)
%input:      number = any number (positive/negative)
%output:     numberabs = absolute of a number
  if anynumber < 0
  numberabs=-1.*anynumber; %transf a neg into pos
  else
  numberabs=anynumber;
  end
end
	\end{verbatim}
\end{exampleblock}	
%\end{block}
\vspace{-5pt}
\begin{block}{}
	Now in Matlab type the commands \verb|lookfor Absolute| vs \verb|Help Absolute|. What is the difference?
\end{block}
		
\end{frame}


%*****************
\begin{frame}[fragile]
	
	\frametitle{\secname : \subsecname}
	%\centering
	\vspace{-7pt}
	\begin{minipage}[t]{0.43\linewidth}
	\vspace{5pt}
	\centering
	\begin{minipage}{0.7\linewidth}
			\begin{block}{}
				Solve: $\sum_{i=1}^{10} \left|i-5\right|$
			\end{block}
	\end{minipage}
	\vspace{5pt}
	\begin{block}{script}
	\textbf{File:} \verb|scriptNofunction.m|
	\vspace{-7pt}
	\begin{verbatim}
	sum=0
	for i=1:10
	    i
	    if i-5<0
	        sum=sum-1*(i-5)
	    else
	        sum=sum+(i-5)
	    end
	end
	\end{verbatim}
	\end{block}
	\end{minipage}
	\hspace{5pt}
	\begin{minipage}[t]{0.53\linewidth}
%	\onslide<2->
	\vspace{-17pt}
	\begin{exampleblock}{Function (No Helpcomments used)}
	\textbf{File:} \verb|Absolute.m|
	\vspace{-7pt}	
	\begin{verbatim}
	function [out] = Absolute (in)
	    if in < 0
	        out=-1*in
	    else
	        out=in
	    end
	end   
	\end{verbatim}
	\end{exampleblock}
	\vspace{-1pt}
	\textbf{File:} \verb|scriptYesfunction.m|
	\vspace{-5pt}	
	\begin{verbatim}
	sum=0
	for i=1:10
	%This calls Function Absolute
	    sum=sum+Abs(i-5)
	end
	\end{verbatim}	
	\end{minipage}
	
\end{frame}


%*****************
\begin{frame}[fragile]
	
	\frametitle{\secname : \subsecname}
	%\centering
	\vspace{-5pt}
	If a function has more than one output, you can save both by using an array with matrices variable in the call. 	
	
	\textbf{File:} \verb|MySum.m|
	\vspace{-3pt}
	\begin{exampleblock}{Function}
		\vspace{-1pt}
		\begin{verbatim}
		function [theArray theSum] = MySum(start,final)
		count=1; theSum = 0
		for i=start:final
		    theSum=theSum+i
		    theArray(count)=i
		    count=count+1
		end
		\end{verbatim}
	\end{exampleblock}
	
	\textbf{File:} \verb|lazysum.m|	
	\vspace{-1pt}
	\begin{block}{script}
		\vspace{-1pt}
		\begin{verbatim}
		%lazy to make the sum everytime?, call your function
		[theArray theSum] = MySum(3,5)
		\end{verbatim}	
	\end{block}
	

	\begin{minipage}[t]{0.43\linewidth}
	\hspace{5pt}	
	\end{minipage}
	
\end{frame}

\section{MATLAB - \href{http://www.mathworks.com/help/matlab/ref/plot.html}{plot} and \href{http://www.mathworks.com/help/matlab/ref/fplot.html}{fplot}}

\subsection{Definition}

%*****************
\begin{frame}[fragile]
	
	\frametitle{\secname : \subsecname}
	%\centering
	\vspace{-17pt}
	\begin{minipage}[t]{0.47\linewidth}
		\begin{block}{}
		%\vspace{-5pt}
		\textbf{Syntax}
		\vspace{-5pt}
		\begin{verbatim}
		plot(X,Y)
		\end{verbatim}
		%\vspace{-5pt}
		creates a 2-D line plot of the data in Y versus the corresponding values in X.
		\end{block}
		\vspace{-10pt}
		\begin{verbatim}
		x=0:10
		y=x.^2
		plot(x,y)
		\end{verbatim}
		\begin{verbatim}
		x=0:10
		y=x.^2
		plot(x,y)
		hold on
		plot(x,x)
		\end{verbatim}		
	\end{minipage}
	\hspace{7pt}
	\begin{minipage}[t]{0.47\linewidth}
	%\onslide<2->
		\begin{block}{}	
		\textbf{Syntax}
		\vspace{-5pt}
		\begin{verbatim}
		fplot(fun,limits) 
		\end{verbatim}
		plots a function between specified limits. \verb|limits| is a vector specifying the x-axis limits \verb|([xmin xmax])|.
	\end{block}
		\vspace{-10pt}
		\begin{verbatim}
		fplot(@Abs, [-10 10])
		% Check the values around 0
		fplot(@Abs, [-10 10], 100)	
		% What happened to the plot?	
		\end{verbatim}
		\vspace{-10pt}
		\begin{exampleblock}{}
		Notice the use of our earlier
		\verb|function [out] = Abs (in);|
		\end{exampleblock}				
	\end{minipage}
	
\end{frame}

\subsection{Examples}
%*****************
\begin{frame}[fragile]
	
	\frametitle{\secname : \subsecname}
	%\centering
	\vspace{-17pt}
	\begin{minipage}[t]{0.47\linewidth}
		\textbf{File:} \verb|plotgas.m|
		\begin{verbatim}
		R = 8.314; 
		T1 = 298;
		T2 = 398;
		T3 = 498;
		% Create an array of Volume 
		% values from 0.01 to 0.1
		%V = linspace(0.01,0.1,100);
		V = 0.01:0.001:0.1;
		% Fill the data arrays
		P1 = Pressure(T1,V,R);
		P2 = Pressure(T2,V,R);
		P3 = Pressure(T3,V,R);
		\end{verbatim}	
	\end{minipage}
	\hspace{7pt}
	\begin{minipage}[t]{0.47\linewidth}
		%\onslide<2->
		\begin{verbatim}
		%continuation
		plot(V,P1,V,P2,V,P3)
		title('Ideal Gas Plot')
		xlabel('Molar Volume')
		ylabel('Pressure')
		legend('T=298','T=398','T=498')
		\end{verbatim}
	\begin{block}{Exercise}
	*Change the value of \verb|V = 0.01:0.0001:0.1;| \\
	*What does this definition do:? \verb|V = linspace(0.01,0.1,100);|
	\end{block}
	\begin{exampleblock}{}
	Notice the use of our earlier function
	\verb|P = Pressure(T,V,R);|
	\end{exampleblock}			
	\end{minipage}
	
\end{frame}


%*****************
\begin{frame}[fragile]
	
	\frametitle{\secname : \subsecname}
	%\centering
	\vspace{-17pt}
	\begin{minipage}[t]{0.53\linewidth}
		\begin{block}{}
		Compare to \verb|plotgas.m|
		\end{block}	
		\textbf{File:} \verb|plotgasalt.m|
		\begin{verbatim}
		close all;
		p1 = @(x) Pressure(298,x,8.314);
		fplot(p1,[0.01 0.1],'g-')
		hold on
		p1 = @(x) Pressure(398,x,8.314);
		fplot(p1,[0.01 0.1],'r:')
		p1 = @(x) Pressure(498,x,8.314);
		fplot(p1,[0.01 0.1],'b')
		\end{verbatim}
	\begin{exampleblock}{}
	Notice the use of our earlier function
	\verb|P = Pressure(T,V,R);|
	\end{exampleblock}			
	\end{minipage}
	\hspace{7pt}
	\begin{minipage}[t]{0.43\linewidth}
		%\onslide<2->
	\begin{block}{}
	Notice the use of
	\verb|'g-'| and \verb|'r:'| and \verb|'b'| 
	\end{block}

	The first letter in the code sets the color of the line.
	\vspace{10pt}
	
	\centering
	\begin{tabular}{c c}
	\hline
	Symbol & Color \\ \hline
		g & green \\
		b & blue \\
		r & red \\
		c & cyan \\
		m & magenta \\
		y & yellow \\
		k & black \\
		w & white\\ 
	\end{tabular} 		
	\end{minipage}
	
\end{frame}


%*****************
\begin{frame}[fragile]
	
	\frametitle{\secname : \subsecname}
	%\centering
	\vspace{-17pt}

The next symbol is the Line Style
\vspace{10pt}
	\centering
	\begin{tabular}{c c}
	\hline
	Symbol & Line Style \\ \hline
	\verb|-|      &  solid line \\
	\verb|--| & dashed line \\
	\verb|:| & dotted line \\
	\verb|-.| & dash-dot line \\
	\end{tabular} 

You can add other symbols to your plot, called marker for the line.
\vspace{10pt}

\begin{tabular}{ c c c c }
\hline
Symbol & Marker Style & Symbol & Marker Style \\ \hline
. & point & v & triangle (down) \\
x & cross & \string^ & triangle (up) \\
+ & plus  & $<$ & triangle (left) \\
s & square & $>$ & triangle (right) \\
d & diamond & * & asterisk \\
p & pentagram & h & hexagram \\
\end{tabular} 
	
\end{frame}


\section{MATLAB - \href{http://www.mathworks.com/help/matlab/ref/subplot.html}{subplot}}
\subsection{Definition}
%*****************
\begin{frame}[fragile]
	
	\frametitle{\secname : \subsecname}
	%\centering
	\vspace{-17pt}
	\begin{minipage}[t]{0.55\linewidth}
	\begin{exampleblock}{}
	\textbf{Syntax:}\\
	\verb|subplot(m,n,p)| \\
	divides the current figure into an \verb|m-by-n| grid and creates an axes for a subplot in the position specified by \verb|p|.
	\end{exampleblock}
	\vspace{-10pt}
	\begin{verbatim}
x = linspace(0,10);

figure(1)
subplot(2,1,1);
p1 = @(x) Pressure(298,x,8.314);
fplot(p1,[0.01 0.1],'g-x')
	
subplot(2,1,2);
p1 = @(x) Pressure(398,x,8.314);
fplot(p1,[0.01 0.1],'r:d')
	\end{verbatim}			
	\end{minipage}
	\hspace{7pt}
	\begin{minipage}[t]{0.41\linewidth}
	\vspace{10pt}
		\textbf{File:} \verb|plotsubplot.m|
		\begin{verbatim}
		clc; clear; close all;
		
		x=0:10;
		y=x.^2;
		subplot(2,2,1)
		plot(x,y)
		
		subplot(2,2,2)
		plot(x,y)
		
		subplot(2,2,[3 4])
		fplot(@Abs, [-10 10])
		\end{verbatim}	
	\end{minipage}
	
\end{frame}

\section{Other commands to remember}
\subsection{Exercises}

%*****************
\begin{frame}[fragile]
\frametitle{\secname}

\begin{exampleblock}{Type this in the command window}
\begin{verbatim}
>> help func    %What does this command do?
>> who          %What does this command do?
>> whos         %What is the difference with >>who ?
>> zeros (3,4)  %What does this command do?
>> ones (5,2)   %What does this command do?
>> linspace(1,10,100) %Compare to >> 1:0.1:10
>> logspace(1,10,100)
>> M=[[1,2,3]'[4,5,6]'[7,8,9]']  %What does command do?
>> log(M)                        %What does command do?
>> length(M)                     %What does command do?
\end{verbatim}
\end{exampleblock}	
\begin{alertblock}{Really look up for what these command do}
	\verb|length(), logspace(), linspace(), hold on|
\end{alertblock}
	
\end{frame}

%*****************
\begin{frame}[fragile]
	\frametitle{\secname}
	
\begin{lstlisting}[caption = {Concatenating text}]
n='Dark '
l='lord'
donotsayit = [n l]
isaidno = ['Dark ' 'lord'] %compare to donotsayit
\end{lstlisting}

\begin{lstlisting}[caption = {Making Tables}]
a=[20; 21; 22];  
b={'Eng'; 'Sci'; 'Sci'} %Notice the {} instead of [] 
X=table(a, b, 'VariableNames',{'age','Major'})
\end{lstlisting}

\begin{block}{}
	Test these commands/script in your computer
\end{block}

\end{frame}

\placelogotrue
%
\begin{frame}
\frametitle{See you next class}
\vspace{-25pt}

\textbf{\textit{``Just as there is not royal road to geometry, there is no royal road to programming''}}.- Euclid and J. V. Guttag
\vspace{7pt}

\textit{The computer will do what you TELL them to do NOT what you WANT them to do}.- Someone in the Internet (Perhaps)
\vspace{7pt}	

\textit{Think twice, code once}.- Anonymous
\vspace{7pt}

\textit{The sooner you start to code, the longer the program will take}.- R. Carlson\vspace{7pt}

\textit{Any fool can write code that a computer can understand. Good programmers write code that humans can understand}.- M. Fowler
\vspace{7pt}

\textit{Simplicity is the soul of efficiency}.- A. Freeman
\vspace{7pt}

\textit{If you cannot grok the overall structure of a program while taking a shower, you are not ready to code it}.- R. Pattis

\end{frame}


\placelogofalse


\section{Appendix: Scripts included}

%\subsection{}

\begin{frame}
\frametitle{\secname}

\vspace{-7pt}
\lstinputlisting[caption={exercise\_inputtypes.m}]{N2_Notes_on_programming_functions/exercise_inputtypes.m}

\vspace{-7pt}
\begin{exampleblock}{}
	Try these commands in your own workstation, i.e. have the lectures on one half side of your screen and Matlab/Octave-GUI on the other half. This is the best approach to learning this.   
\end{exampleblock}

\begin{alertblock}{}
	Check the scripts/functions under the directory for this note number (X): \newline
	 /NX\_Notes\_directory
\end{alertblock}

\end{frame}	


\end{document}